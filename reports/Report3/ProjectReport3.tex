\documentclass[11pt,]{article}
\usepackage{lmodern}
\usepackage{amssymb,amsmath}
\usepackage{ifxetex,ifluatex}
\usepackage{fixltx2e} % provides \textsubscript
\ifnum 0\ifxetex 1\fi\ifluatex 1\fi=0 % if pdftex
  \usepackage[T1]{fontenc}
  \usepackage[utf8]{inputenc}
\else % if luatex or xelatex
  \ifxetex
    \usepackage{mathspec}
  \else
    \usepackage{fontspec}
  \fi
  \defaultfontfeatures{Ligatures=TeX,Scale=MatchLowercase}
\fi
% use upquote if available, for straight quotes in verbatim environments
\IfFileExists{upquote.sty}{\usepackage{upquote}}{}
% use microtype if available
\IfFileExists{microtype.sty}{%
\usepackage{microtype}
\UseMicrotypeSet[protrusion]{basicmath} % disable protrusion for tt fonts
}{}
\usepackage[margin=1in]{geometry}
\usepackage[unicode=true]{hyperref}
\PassOptionsToPackage{usenames,dvipsnames}{color} % color is loaded by hyperref
\hypersetup{
            pdftitle={Project Report 3},
            colorlinks=true,
            linkcolor=Maroon,
            citecolor=blue,
            urlcolor=blue,
            breaklinks=true}
\urlstyle{same}  % don't use monospace font for urls
\setlength{\emergencystretch}{3em}  % prevent overfull lines
\providecommand{\tightlist}{%
  \setlength{\itemsep}{0pt}\setlength{\parskip}{0pt}}
\setcounter{secnumdepth}{0}
% Redefines (sub)paragraphs to behave more like sections
\ifx\paragraph\undefined\else
\let\oldparagraph\paragraph
\renewcommand{\paragraph}[1]{\oldparagraph{#1}\mbox{}}
\fi
\ifx\subparagraph\undefined\else
\let\oldsubparagraph\subparagraph
\renewcommand{\subparagraph}[1]{\oldsubparagraph{#1}\mbox{}}
\fi

% set default figure placement to htbp
\makeatletter
\def\fps@figure{htbp}
\makeatother



% Stuff I added.
% --------------

\usepackage{indentfirst}
\usepackage[doublespacing]{setspace}
\usepackage{fancyhdr}
\pagestyle{fancy}
\usepackage{layout}   
\lhead{\sc Project Report 3}
\chead{}
\rhead{\thepage}
\lfoot{}
\cfoot{}
\rfoot{}

\renewcommand{\headrulewidth}{0.0pt}
\renewcommand{\footrulewidth}{0.0pt}

\usepackage{sectsty}
\sectionfont{\centering}
\subsectionfont{\centering}

\newtheorem{hypothesis}{Hypothesis}

% Begin document
% --------------

\begin{document}

\doublespacing

\begin{titlepage}
    \begin{center}
    \line(1,0){300} \\ 
    [0.25in]
    \huge{\bfseries Project Report 3} \\
    [2mm]
    \line(1,0){200} \\
    [1.5cm] 
    \textsc{\Large Fault Localization using NLP on Bug Reports} \\
    [0.75cm]
    \textsc{\Large with Kostas Kontogiannis} \\
    [10cm]
    \end{center}
    
    \begin{flushright}
    \textsc{\Large{Gurpreet Singh \& Paul Bartlett \\} \normalsize\emph{Software Developers \\} }
    
    \end{flushright}
    
\end{titlepage}


\newpage

{
\hypersetup{linkcolor=black}
\setcounter{tocdepth}{2}
\tableofcontents
\newpage
}
\hypertarget{faulty}{%
\section{Faulty}\label{faulty}}

\hypertarget{project-description}{%
\subsection{Project description}\label{project-description}}

The focus will be to design and develop a system that is capable of
processing bug reports and extracting useful information about them, and
then using that information to provide the developers useful insight
into where the bug may be within a large code base. The goal is to
reduce the amount of time a developer will need to reach the correct bug
after initially reviewing the bug report.

The system will be developed in such a way that it is easily integrated
into a continuous delivery pipeline. The project can be divided into
three distinct components.

\hypertarget{data-processing}{%
\subsubsection{Data Processing}\label{data-processing}}

In order for the entire system to work correctly the core essential data
processing and analysis has to be effective in detecting the errors.
Therefore, the first step is developing a system that can use NLP to
process all the bug reports associated with a project to come up with a
list of keywords and process the code base to determine a map of
relationships between function calls. This code has now been written and
is mostly within our project.

\hypertarget{user-interface}{%
\subsubsection{User Interface}\label{user-interface}}

The next step in delivering the system to a real user, is developing a
front end where a user can input a repository for the system to begin
processing. Ideal operation of this tool would occur like other DevOps
pipeline tools such as travis-ci.org where a user can link a repository
they own and the tool can push it's results back into the bug report for
developers to see. There will not be too many interactions available on
the front end other than viewing the results of the system and picking
new repositories. This portion of the project is not yet completed.
Refer to deviations section to understand the reasoning for putting the
interface as the last priority.

\hypertarget{runtime-processing}{%
\subsubsection{Runtime Processing}\label{runtime-processing}}

Since the front end will be making REST API calls to Github
repositories, and we need a way to persist processing while providing
consistent feedback to users, there needs to be a backend API service
allowing those operations to occur. Another task this portion will be
responsible for is handling the flow of information when a new bug
report appears. The backend will be responsible for detecting this,
starting a new processing task, and posting a ``Fault Report'' back into
the bug discussion. A substantial amount of work has been done on this
portion of the project. The backend is able to communicate with Github
and catch issues posted by users.

\hypertarget{revised-roles}{%
\subsection{Revised Roles}\label{revised-roles}}

\textbf{Gurpreet Singh}

\begin{itemize}
\item
  Lead Architect / Back-End developer
\item
  Documenter
\end{itemize}

\textbf{Paul Bartlett}

\begin{itemize}
\item
  Lead Front-End developer
\item
  Lead Tester \& Quality Controller
\end{itemize}

\hypertarget{revised-system-requirements}{%
\subsection{Revised System
Requirements}\label{revised-system-requirements}}

\hypertarget{section-a-data-processing}{%
\subsubsection{Section A : Data
Processing}\label{section-a-data-processing}}

\begin{itemize}
\tightlist
\item
  \textbf{Feature 1:} Able to generate entity relationship rsf from
  codebase

  \begin{itemize}
  \tightlist
  \item
    FR 1: Pass code through cdif2rsf to generate rsf
  \item
    FR 2: Clean up incorrect entity and relationships
  \item
    FR 3: Store in accessible data storage for next step to use
  \end{itemize}
\item
  \textbf{Feature 2:} Able to generate set of keywords from bug
  description

  \begin{itemize}
  \tightlist
  \item
    FR 1: Compare each token to codebase to find valid functions
  \item
    FR 2: Expand initial token set by a factor of 3
  \item
    FR 3: Use NLP to determine question context
  \end{itemize}
\item
  \textbf{Feature 3:} Able to combine keywords and rsf into ranked
  outcomes

  \begin{itemize}
  \tightlist
  \item
    FR 1: Run LSI on each token and generate search space for each
  \item
    FR 2: Expand the search space for each result in FR 1
  \item
    FR 3: Find similarities between the initial token expansion and the
    final set of tokens
  \item
    FR 4: Apply ranking equation from research paper to come up with
    final outcome
  \end{itemize}
\end{itemize}

\hypertarget{section-b-front-end-user-interface}{%
\subsubsection{Section B : Front-End User
Interface}\label{section-b-front-end-user-interface}}

\begin{itemize}
\tightlist
\item
  \textbf{Feature 4:} User is able to scan and mark a new repository for
  processing

  \begin{itemize}
  \tightlist
  \item
    FR 1: Scan user's Github repos using Github's API
  \item
    FR 2: Allow the user to select ones they wish to run processing on
  \item
    FR 3: Remember which ones the user selected by storing on backend
  \end{itemize}
\item
  \textbf{Feature 5:} User is able to view the results of a new bug
  report's processing

  \begin{itemize}
  \tightlist
  \item
    FR 1: Monitor output from backend endpoints showing new results for
    user's repos
  \item
    FR 2: When a new bug report is created, and the processing finishes,
    show the output of that processing on a separate page
  \item
    FR 3: Allow the user to rerun processing on a specific bug report by
    sending a request to backend
  \end{itemize}
\item
  \textbf{Feature 6:} User is able to login using their Github
  credentials

  \begin{itemize}
  \tightlist
  \item
    FR 1: On first usage redirect user to Github's App authentication
    page
  \item
    FR 2: Ask backend to associate bug reports and repositories with
    this user
  \item
    FR 3: Redirect user to main UI
  \end{itemize}
\end{itemize}

\hypertarget{section-c-back-end-runtime-processing}{%
\subsubsection{Section C : Back-End Runtime
Processing}\label{section-c-back-end-runtime-processing}}

\begin{itemize}
\tightlist
\item
  \textbf{Feature 7:} Support front-end operations

  \begin{itemize}
  \tightlist
  \item
    FR 1: Allow registration using Github Auth
  \item
    FR 2: Allow retrieval of processing results for each bug report
  \item
    FR 3: Support re-running processing on a specific bug report
  \item
    FR 4: Create a API where the UI can fetch everything from
  \end{itemize}
\item
  \textbf{Feature 8:} Manage the automation of Data Processing (F1, F2,
  and F3)

  \begin{itemize}
  \tightlist
  \item
    FR 1: Automate RSF generation when a new repository is linked
  \item
    FR 2: Automate keyword generation when a new repository is linked
  \item
    FR 3: Continuously improve and modify RSF and keywords as code/bugs
    change
  \end{itemize}
\item
  \textbf{Feature 9:} Handle processing and evaluation when a new bug
  report comes in

  \begin{itemize}
  \tightlist
  \item
    FR 1: Monitor marked repositories for each registered user and
    trigger when new issue is filed
  \item
    FR 2: Run through automated ranking algorithm
  \item
    FR 3: Store result for later retrieval
  \item
    FR 4: Be able to connect to a repository and fetch an issue when it
    is posted
  \end{itemize}
\item
  \textbf{Feature 10:} Combine each element of data processing into the
  backend runtime

  \begin{itemize}
  \tightlist
  \item
    FR 1: Combine entity generation into token generation
  \item
    FR 2: Combine FR1 with RSF processing all into one unit
  \item
    FR 3: Move all functionality to an exposed part of the API
  \item
    FR 4: Create all endpoints for data processing functionality
  \end{itemize}
\end{itemize}

\hypertarget{project-actual-and-deviations}{%
\subsection{Project Actual and
Deviations}\label{project-actual-and-deviations}}

Refer to the attached spreadsheet to see the current progress on the
project. Some deviations we made from the original plan during this
iteration include changing the priority of some features and spending
more time on parts of the project we thought would be easy. One of the
causes for this was the lack of code organization and commenting present
in the research project we are working with. This lead to a lot of
reformatting which wasn't terrible but still consumed time we would have
liked to spend elsewhere. We were also relying on receiving an
additional research code base to reference but we were unable to
communicate with the people in charge of the research.

We changed the priority of the frontend features and made them last
priority. This was done because we decided that having the functional
portion of the code working is much more important than focusing on the
looks.

\hypertarget{list-of-features-that-have-been-implemented}{%
\subsubsection{List of features that have been
implemented}\label{list-of-features-that-have-been-implemented}}

Feature 1

\begin{itemize}
\tightlist
\item
  Requirement 1
\item
  Requirement 2
\end{itemize}

Feature 2

\begin{itemize}
\tightlist
\item
  Requirement 1
\item
  Requirement 2
\end{itemize}

Feature 3

\begin{itemize}
\tightlist
\item
  Requirement 1
\item
  Requirement 2
\item
  Requirement 3
\item
  Requirement 4
\end{itemize}

Feature 7

\begin{itemize}
\tightlist
\item
  Requirement 1
\item
  Requirement 2
\item
  Requirement 4
\end{itemize}

Feature 9

\begin{itemize}
\tightlist
\item
  Requirement 1
\item
  Requirement 4
\end{itemize}

Feature 10

\begin{itemize}
\tightlist
\item
  Requirement 4
\end{itemize}

\hypertarget{future-implementation-plan-to-completion}{%
\subsection{Future implementation plan to
completion}\label{future-implementation-plan-to-completion}}

The next step for this project is putting everything together. By the
next milestone we want to atleast be able to completely aggregate issues
from a Github repo and produce results to the backend. When this process
is reliable and consistent we will create a minimal web interface to
display the results.

The next major problem we will be running into is combining multiple
parts of the data processing section (features 1, 2, 3) into the runtime
section (features 7, 8, 9). Currently they are operating separately of
each other. We have created a new feature (10) to handle work related to
this task.

\end{document}

