\documentclass[ignorenonframetext,]{beamer}
\setbeamertemplate{caption}[numbered]
\setbeamertemplate{caption label separator}{: }
\setbeamercolor{caption name}{fg=normal text.fg}
\beamertemplatenavigationsymbolsempty
\usepackage{lmodern}
\usepackage{amssymb,amsmath}
\usepackage{ifxetex,ifluatex}
\usepackage{fixltx2e} % provides \textsubscript
\ifnum 0\ifxetex 1\fi\ifluatex 1\fi=0 % if pdftex
  \usepackage[T1]{fontenc}
  \usepackage[utf8]{inputenc}
\else % if luatex or xelatex
  \ifxetex
    \usepackage{mathspec}
  \else
    \usepackage{fontspec}
  \fi
  \defaultfontfeatures{Ligatures=TeX,Scale=MatchLowercase}
\fi
% use upquote if available, for straight quotes in verbatim environments
\IfFileExists{upquote.sty}{\usepackage{upquote}}{}
% use microtype if available
\IfFileExists{microtype.sty}{%
\usepackage{microtype}
\UseMicrotypeSet[protrusion]{basicmath} % disable protrusion for tt fonts
}{}
\newif\ifbibliography
\hypersetup{
            pdftitle={Faulty},
            pdfauthor={Gurpreet Singh},
            pdfborder={0 0 0},
            breaklinks=true}
\urlstyle{same}  % don't use monospace font for urls

% Prevent slide breaks in the middle of a paragraph:
\widowpenalties 1 10000
\raggedbottom

\AtBeginPart{
  \let\insertpartnumber\relax
  \let\partname\relax
  \frame{\partpage}
}
\AtBeginSection{
  \ifbibliography
  \else
    \let\insertsectionnumber\relax
    \let\sectionname\relax
    \frame{\sectionpage}
  \fi
}
\AtBeginSubsection{
  \let\insertsubsectionnumber\relax
  \let\subsectionname\relax
  \frame{\subsectionpage}
}

\setlength{\parindent}{0pt}
\setlength{\parskip}{6pt plus 2pt minus 1pt}
\setlength{\emergencystretch}{3em}  % prevent overfull lines
\providecommand{\tightlist}{%
  \setlength{\itemsep}{0pt}\setlength{\parskip}{0pt}}
\setcounter{secnumdepth}{0}

\title{Faulty}
\subtitle{Bug Localization as a Service}
\author{Gurpreet Singh}
\date{2018-04-09}

\begin{document}
\frame{\titlepage}

\subsection{Background and Related
Work}\label{background-and-related-work}

\begin{frame}{Preprocessing scripts}

\begin{itemize}
\tightlist
\item
  Extracted rsf from code
\item
  Formatted bug reports for processing
\item
  Prepares file structure for the BugLocalization project
\end{itemize}

\end{frame}

\begin{frame}{BugLocalization}

\begin{itemize}
\tightlist
\item
  Used extracted bug reports and code rsf
\item
  Produces output ranking files based on probability of containing a bug
\item
  Written in Shell, Python and Java
\item
  User interface was command line based with minimal input
\item
  Only supported single operations
\end{itemize}

\end{frame}

\subsection{Concepts}\label{concepts}

\begin{frame}{RSF}

\begin{itemize}
\tightlist
\item
  A map of relationships betweens tokens within a codebase
\item
  A token is a keyword in the codebase (ex. method)
\item
  Created using preprocessing scripts
\item
  Allows us to verify tokens inside of bug reports
\end{itemize}

\end{frame}

\begin{frame}{Bug Report}

\begin{itemize}
\tightlist
\item
  A database entry for each bug report
\item
  Contains a String field containing what a user wrote inside their bug
  report
\end{itemize}

\end{frame}

\begin{frame}{Token Expansion}

\begin{itemize}
\tightlist
\item
  Tokens extracted from the bug report can be expanded
\item
  Expansion means including tokens that are referenced by the original
  token set
\end{itemize}

\end{frame}

\subsection{Development Objectives}\label{development-objectives}

\begin{frame}{Develop a complete system}

\begin{itemize}
\tightlist
\item
  Combine scripts and java code into a easy to access complete system
\item
  System should be able to handle new codebases and improve with new bug
  reports
\end{itemize}

\end{frame}

\begin{frame}{Be able to integrate with a Github based workflow}

\begin{itemize}
\tightlist
\item
  Fetch new issues are they are created in a repo and process them
\item
  Push results back to the issue page so the developers can get a head
  start
\item
  Allow users to authenticate using Github
\end{itemize}

\end{frame}

\begin{frame}{Operate as a standalone service with a UI}

\begin{itemize}
\tightlist
\item
  Work in a similar fashion to other CI tools (like travis, etc)
\item
  Allow users control to hook into any of their repositories
\end{itemize}

\end{frame}

\end{document}
