\documentclass[11pt,]{article}
\usepackage{lmodern}
\usepackage{amssymb,amsmath}
\usepackage{ifxetex,ifluatex}
\usepackage{fixltx2e} % provides \textsubscript
\ifnum 0\ifxetex 1\fi\ifluatex 1\fi=0 % if pdftex
  \usepackage[T1]{fontenc}
  \usepackage[utf8]{inputenc}
\else % if luatex or xelatex
  \ifxetex
    \usepackage{mathspec}
  \else
    \usepackage{fontspec}
  \fi
  \defaultfontfeatures{Ligatures=TeX,Scale=MatchLowercase}
\fi
% use upquote if available, for straight quotes in verbatim environments
\IfFileExists{upquote.sty}{\usepackage{upquote}}{}
% use microtype if available
\IfFileExists{microtype.sty}{%
\usepackage{microtype}
\UseMicrotypeSet[protrusion]{basicmath} % disable protrusion for tt fonts
}{}
\usepackage[margin=1in]{geometry}
\usepackage[unicode=true]{hyperref}
\PassOptionsToPackage{usenames,dvipsnames}{color} % color is loaded by hyperref
\hypersetup{
            pdftitle={Project Report 2},
            colorlinks=true,
            linkcolor=Maroon,
            citecolor=blue,
            urlcolor=blue,
            breaklinks=true}
\urlstyle{same}  % don't use monospace font for urls
\setlength{\emergencystretch}{3em}  % prevent overfull lines
\providecommand{\tightlist}{%
  \setlength{\itemsep}{0pt}\setlength{\parskip}{0pt}}
\setcounter{secnumdepth}{0}
% Redefines (sub)paragraphs to behave more like sections
\ifx\paragraph\undefined\else
\let\oldparagraph\paragraph
\renewcommand{\paragraph}[1]{\oldparagraph{#1}\mbox{}}
\fi
\ifx\subparagraph\undefined\else
\let\oldsubparagraph\subparagraph
\renewcommand{\subparagraph}[1]{\oldsubparagraph{#1}\mbox{}}
\fi

% set default figure placement to htbp
\makeatletter
\def\fps@figure{htbp}
\makeatother



% Stuff I added.
% --------------

\usepackage{indentfirst}
\usepackage[doublespacing]{setspace}
\usepackage{fancyhdr}
\pagestyle{fancy}
\usepackage{layout}   
\lhead{\sc Project Report 2}
\chead{}
\rhead{\thepage}
\lfoot{}
\cfoot{}
\rfoot{}

\renewcommand{\headrulewidth}{0.0pt}
\renewcommand{\footrulewidth}{0.0pt}

\usepackage{sectsty}
\sectionfont{\centering}
\subsectionfont{\centering}

\newtheorem{hypothesis}{Hypothesis}

% Begin document
% --------------

\begin{document}

\doublespacing

\begin{titlepage}
    \begin{center}
    \line(1,0){300} \\ 
    [0.25in]
    \huge{\bfseries Project Report 2} \\
    [2mm]
    \line(1,0){200} \\
    [1.5cm] 
    \textsc{\Large Fault Localization using NLP on Bug Reports} \\
    [0.75cm]
    \textsc{\Large with Kostas Kontogiannis} \\
    [10cm]
    \end{center}
    
    \begin{flushright}
    \textsc{\Large{Gurpreet Singh \& Paul Bartlett \\} \normalsize\emph{Software Developers \\} }
    
    \end{flushright}
    
\end{titlepage}


\newpage

{
\hypersetup{linkcolor=black}
\setcounter{tocdepth}{2}
\tableofcontents
\newpage
}
\hypertarget{faulty}{%
\section{Faulty}\label{faulty}}

\hypertarget{project-description}{%
\subsection{Project description}\label{project-description}}

The focus will be to design and develop a system that is capable of
processing bug reports and extracting useful information about them, and
then using that information to provide the developers useful insight
into where the bug may be within a large code base. The goal is to
reduce the amount of time a developer will need to reach the correct bug
after initially reviewing the bug report.

The system will be developed in such a way that it is easily integrated
into a continuous delivery pipeline. The project can be divided into
three distinct components.

\hypertarget{data-processing}{%
\subsubsection{Data Processing}\label{data-processing}}

In order for the entire system to work correctly the core essential data
processing and analysis has to be effective in detecting the errors.
Therefore, the first step is developing a system that can use NLP to
process all the bug reports associated with a project to come up with a
list of keywords and process the code base to determine a map of
relationships between function calls.

\hypertarget{user-interface}{%
\subsubsection{User Interface}\label{user-interface}}

The next step in delivering the system to a real user, is developing a
front end where a user can input a repository for the system to begin
processing. Ideal operation of this tool would occur like other DevOps
pipeline tools such as Travis.io where a user can link a repository they
own and the tool can push it's results back into the bug report for
developers to see. There will not be too many interactions available on
the front end other than viewing the results of the system and picking
new repositories.

\hypertarget{runtime-processing}{%
\subsubsection{Runtime Processing}\label{runtime-processing}}

Since the front end will be making REST API calls to Github
repositories, and we need a way to persist processing while providing
consistant feedback to users, there needs to be a backend API service
allowing those operations to occur. Another task this portion will be
responsible for is handling the flow of information when a new bug
report appears. The backend will be responsible for detecting this,
starting a new processing task, and posting a ``Fault Report'' back into
the bug discussion.

\hypertarget{roles}{%
\subsection{Roles}\label{roles}}

\textbf{Gurpreet Singh}

\begin{itemize}
\item
  Lead Architect
\item
  Documenter
\end{itemize}

\textbf{Paul Bartlett}

\begin{itemize}
\item
  Project Manager
\item
  Lead Requirements Analyst
\item
  Lead Tester \& Quality Controller
\end{itemize}

\hypertarget{revised-system-requirements}{%
\subsection{Revised System
Requirements}\label{revised-system-requirements}}

\hypertarget{section-a-data-processing}{%
\subsubsection{Section A : Data
Processing}\label{section-a-data-processing}}

\begin{itemize}
\tightlist
\item
  \textbf{Feature 1:} Able to generate entity relationship rsf from
  codebase

  \begin{itemize}
  \tightlist
  \item
    FR 1: Pass code through cdif2rsf to generate rsf
  \item
    FR 2: Clean up incorrect entity and relationships
  \item
    FR 3: Store in accessible data storage for next step to use
  \end{itemize}
\item
  \textbf{Feature 2:} Able to generate set of keywords from bug
  description

  \begin{itemize}
  \tightlist
  \item
    FR 1: Compare each token to codebase to find valid functions
  \item
    FR 2: Expand initial token set by a factor of 3
  \item
    FR 3: Use NLP to determine question context
  \end{itemize}
\item
  \textbf{Feature 3:} Able to combine keywords and rsf into ranked
  outcomes

  \begin{itemize}
  \tightlist
  \item
    FR 1: Run LSI on each token and generate search space for each
  \item
    FR 2: Expand the search space for each result in FR 1
  \item
    FR 3: Find similarities between the initial token expansion and the
    final set of tokens
  \item
    FR 4: Apply ranking equation from research paper to come up with
    final outcome
  \end{itemize}
\end{itemize}

\hypertarget{section-b-front-end-user-interface}{%
\subsubsection{Section B : Front-End User
Interface}\label{section-b-front-end-user-interface}}

\begin{itemize}
\tightlist
\item
  \textbf{Feature 4:} Doing Doing

  \begin{itemize}
  \tightlist
  \item
    FR 1: Too much doing doing
  \item
    FR 2: Too much doing doing
  \item
    FR 3: Too much doing doing
  \end{itemize}
\item
  \textbf{Feature 5:} Doing Doing

  \begin{itemize}
  \tightlist
  \item
    FR 1: Too much doing doing
  \item
    FR 2: Too much doing doing
  \item
    FR 3: Too much doing doing
  \end{itemize}
\item
  \textbf{Feature 6:} Doing Doing

  \begin{itemize}
  \tightlist
  \item
    FR 1: Too much doing doing
  \item
    FR 2: Too much doing doing
  \item
    FR 3: Too much doing doing
  \end{itemize}
\end{itemize}

\hypertarget{section-c-back-end-runtime-processing}{%
\subsubsection{Section C : Back-End Runtime
Processing}\label{section-c-back-end-runtime-processing}}

\begin{itemize}
\tightlist
\item
  \textbf{Feature 7:} Doing Doing

  \begin{itemize}
  \tightlist
  \item
    FR 1: Too much doing doing
  \item
    FR 2: Too much doing doing
  \item
    FR 3: Too much doing doing
  \end{itemize}
\item
  \textbf{Feature 8:} Doing Doing

  \begin{itemize}
  \tightlist
  \item
    FR 1: Too much doing doing
  \item
    FR 2: Too much doing doing
  \item
    FR 3: Too much doing doing
  \end{itemize}
\item
  \textbf{Feature 9:} Doing Doing

  \begin{itemize}
  \tightlist
  \item
    FR 1: Too much doing doing
  \item
    FR 2: Too much doing doing
  \item
    FR 3: Too much doing doing
  \end{itemize}
\end{itemize}

\hypertarget{project-deviation}{%
\subsection{Project Deviation}\label{project-deviation}}

The project plan has completed changed into a new idea. In the first
report we thought we were going to be working on a system that uses bug
data, test data and other external sources to predict the effect of line
changes in a code base. Our new idea outlined in the project description
above has a much narrower focus and in our opinions a much greater
impact. We are excited to work on this new idea

\end{document}

