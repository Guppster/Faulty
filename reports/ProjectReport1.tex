\documentclass[11pt,]{article}
\usepackage{lmodern}
\usepackage{amssymb,amsmath}
\usepackage{ifxetex,ifluatex}
\usepackage{fixltx2e} % provides \textsubscript
\ifnum 0\ifxetex 1\fi\ifluatex 1\fi=0 % if pdftex
  \usepackage[T1]{fontenc}
  \usepackage[utf8]{inputenc}
\else % if luatex or xelatex
  \ifxetex
    \usepackage{mathspec}
  \else
    \usepackage{fontspec}
  \fi
  \defaultfontfeatures{Ligatures=TeX,Scale=MatchLowercase}
\fi
% use upquote if available, for straight quotes in verbatim environments
\IfFileExists{upquote.sty}{\usepackage{upquote}}{}
% use microtype if available
\IfFileExists{microtype.sty}{%
\usepackage{microtype}
\UseMicrotypeSet[protrusion]{basicmath} % disable protrusion for tt fonts
}{}
\usepackage[margin=1in]{geometry}
\usepackage[unicode=true]{hyperref}
\PassOptionsToPackage{usenames,dvipsnames}{color} % color is loaded by hyperref
\hypersetup{
            pdftitle={Project Report 1},
            colorlinks=true,
            linkcolor=Maroon,
            citecolor=blue,
            urlcolor=blue,
            breaklinks=true}
\urlstyle{same}  % don't use monospace font for urls
\setlength{\emergencystretch}{3em}  % prevent overfull lines
\providecommand{\tightlist}{%
  \setlength{\itemsep}{0pt}\setlength{\parskip}{0pt}}
\setcounter{secnumdepth}{0}
% Redefines (sub)paragraphs to behave more like sections
\ifx\paragraph\undefined\else
\let\oldparagraph\paragraph
\renewcommand{\paragraph}[1]{\oldparagraph{#1}\mbox{}}
\fi
\ifx\subparagraph\undefined\else
\let\oldsubparagraph\subparagraph
\renewcommand{\subparagraph}[1]{\oldsubparagraph{#1}\mbox{}}
\fi

% set default figure placement to htbp
\makeatletter
\def\fps@figure{htbp}
\makeatother



% Stuff I added.
% --------------

\usepackage{indentfirst}
\usepackage[doublespacing]{setspace}
\usepackage{fancyhdr}
\pagestyle{fancy}
\usepackage{layout}   
\lhead{\sc Project Report 1}
\chead{}
\rhead{\thepage}
\lfoot{}
\cfoot{}
\rfoot{}

\renewcommand{\headrulewidth}{0.0pt}
\renewcommand{\footrulewidth}{0.0pt}

\usepackage{sectsty}
\sectionfont{\centering}
\subsectionfont{\centering}

\newtheorem{hypothesis}{Hypothesis}

% Begin document
% --------------

\begin{document}

\doublespacing

\begin{titlepage}
    \begin{center}
    \line(1,0){300} \\ 
    [0.25in]
    \huge{\bfseries Project Report 1} \\
    [2mm]
    \line(1,0){200} \\
    [1.5cm] 
    \textsc{\Large Frameworks to Support Continuous Delivery} \\
    [0.75cm]
    \textsc{\Large with Kostas Kangaro} \\
    [10cm]
    \end{center}
    
    \begin{flushright}
    \textsc{\Large{Gurpreet Singh \& Paul Bartlett \\} \normalsize\emph{\# 250674134 \& 250123456 \\} \normalsize\emph{Software Developers \\} }
    
    \end{flushright}
    
\end{titlepage}


\newpage

{
\hypersetup{linkcolor=black}
\setcounter{tocdepth}{2}
\tableofcontents
\newpage
}
\section{CS4470Y}\label{cs4470y}

\subsection{Project description}\label{project-description}

shortening bug fixing shortening maintenance cycles Achieving continuous
delivery

\subsection{Roles}\label{roles}

\textbf{Gurpreet Singh}

\begin{itemize}
\item
  Lead Architect
\item
  Documenter
\end{itemize}

\textbf{Paul Bartlett}

\begin{itemize}
\item
  Project Manager
\item
  Lead Requirements Analyst
\item
  Lead Tester \& Quality Controller
\end{itemize}

\subsection{System Requirements}\label{system-requirements}

\subsubsection{Section A : Data collection and
modeling}\label{section-a-data-collection-and-modeling}

\begin{itemize}
\item
  \textbf{Feature 1:} Git Based Data Collection
\item
  FR 1: Extract commits from a .git directory
\item
  FR 2: Analyze and record classes/methods effected by each commit
\item
  FR 3: Process and record the amount of code changed with each commit
\item
  \textbf{Feature 2:} Test Based Data Collection
\item
  FR 1: Record results of the test suite each time it is run
\item
  FR 2: During test phase inject previous risk analysis
\item
  \textbf{Feature 3:} Issue Based Data Collection
\item
  FR 1: Record issues filed for the code base
\item
  FR 2: Connect issues to their resolution and make an inference from
  the association
\item
  \textbf{Feature 4:} Data combination
\item
  FR 1: Standardize and combine Git data into a central data model
\item
  FR 2: Standardize and combine test data into a central data model
\item
  FR 3: Standardize and combine issue data into a central data model
\end{itemize}

\subsubsection{Section B : Creating report processing
software}\label{section-b-creating-report-processing-software}

\begin{itemize}
\item
  \textbf{Feature 4:} Conduct Risk Analytics
\item
  FR 1: In the data model figure out risky parts of the code and mark
  them
\item
  FR 2: Identify issue keywords that are higher risk than others
\item
  FR 3: Identify impact when a test is changed (how much coverage it
  offers)
\item
  \textbf{Feature 5:} User Views
\item
  FR 1: Create a PDF report for stakeholders to see progress in codebase
\item
  FR 2: Create a Web view for exploring the risk of the codebase
\item
  FR 3: Create Github, IFTTT and Slack integrations for viewing
  framework data model
\end{itemize}

\subsubsection{Section C : Creating enhancements to dev
environments}\label{section-c-creating-enhancements-to-dev-environments}

\begin{itemize}
\item
  \textbf{Feature 6:} Continuous Integration
\item
  FR 1: Create test integration so developer can see risk rating when
  conducting tests
\item
  FR 2: Develop command line interface to the framework
\item
  FR 3: Research deployment frameworks and integrate risk analysis into
  pipe line
\item
  \textbf{Feature 7:} IDE integration
\item
  FR 1: Develop VIM plugin for viewing risk rating of each method/class
\item
  FR 2: Develop IDEA plugin for viewing risk rating of each method/class
\item
  \textbf{Feature 8:} Microservice Plug-And-Play library
\item
  FR 1: Make the framework easily deployable into a microservice
\item
  FR 2: Reduce dependencies and deliver a minimally intrusive product
\end{itemize}

\end{document}

