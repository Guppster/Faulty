\documentclass[12pt]{article}

%%%%%%%%%%%%%%%%%%%%%%%%%%%%%%%%%%%%%%%%%%%%%%%%%%%%%%%%%%%%%%%%%%%%%
%% Place any additional packages needed here.  Only include packages
%% which are essential, to avoid problems later.
%%%%%%%%%%%%%%%%%%%%%%%%%%%%%%%%%%%%%%%%%%%%%%%%%%%%%%%%%%%%%%%%%%%%%
\usepackage{hyperref}
\usepackage[margin=1in]{geometry}
\usepackage{mathptmx}
\usepackage{fancyhdr}
\usepackage{lastpage}

\setcounter{secnumdepth}{0}% % Turns off numbering for sections
\pagestyle{fancy}
\rhead{ Singh, Gurpreet.  Bartlett, Paul. }
\lhead{\thepage\ of \pageref{LastPage}}

\usepackage{chemformula} % Formula subscripts using \ch{}
\usepackage[T1]{fontenc} % Use modern font encodings
\usepackage[utf8]{inputenc} % Required for inputting international characters
\usepackage[T1]{fontenc} % Output font encoding for international characters

\hypersetup{
    pdftitle={Faulty: Fault Localization as a Service},    % title
    pdfauthor={},     % author
    pdfcreator={},   % creator of the document
}

\providecommand{\tightlist}{%
  \setlength{\itemsep}{0pt}\setlength{\parskip}{0pt}}

%%%%%%%%%%%%%%%%%%%%%%%%%%%%%%%%%%%%%%%%%%%%%%%%%%%%%%%%%%%%%%%%%%%%%
%% If issues arise when submitting your manuscript, you may want to
%% un-comment the next line.  This provides information on the
%% version of every file you have used.
%%%%%%%%%%%%%%%%%%%%%%%%%%%%%%%%%%%%%%%%%%%%%%%%%%%%%%%%%%%%%%%%%%%%%
%%\listfiles

%%%%%%%%%%%%%%%%%%%%%%%%%%%%%%%%%%%%%%%%%%%%%%%%%%%%%%%%%%%%%%%%%%%%%
%% Place any additional macros here.  Please use \newcommand* where
%% possible, and avoid layout-changing macros (which are not used
%% when typesetting).
%%%%%%%%%%%%%%%%%%%%%%%%%%%%%%%%%%%%%%%%%%%%%%%%%%%%%%%%%%%%%%%%%%%%%

\begin{document}

%----------------------------------------------------------------------------------------
%	TITLE PAGE
%----------------------------------------------------------------------------------------

\begin{titlepage} % Suppresses displaying the page number on the title page and the subsequent page counts as page 1
	\newcommand{\HRule}{\rule{\linewidth}{0.5mm}} % Defines a new command for horizontal lines, change thickness here
	
	\center % Centre everything on the page
	
	%------------------------------------------------
	%	Headings
	%------------------------------------------------
	
	\textsc{\LARGE University of Western Ontario}\\[0.8cm] % Main heading such as the name of your university/college

	\textsc{\Large Department of Computer Science}\\[1.5cm] % Major heading such as course name
	
	\textsc{\large CS4470Y: Software Maintenance and Configuration}\\[0.8cm] % Major heading such as course name
	
	\textsc{\large Final Project Report}\\[0.8cm] % Minor heading such as course title
	
	%------------------------------------------------
	%	Title
	%------------------------------------------------
	
	\HRule\\[0.4cm]
	
	{\huge\bfseries Faulty: Fault Localization as a Service}\\[0.4cm] % Title of your document
	
	\HRule\\[1.5cm]
	
	%------------------------------------------------
	%	Author(s)
	%------------------------------------------------
	
	\begin{minipage}{0.4\textwidth}
		\begin{flushleft}
			\large
			\textit{Author}\\
            			Gurpreet \textsc{Singh} \\             			Paul \textsc{Bartlett} \\ 		\end{flushleft}
	\end{minipage}
	~
	\begin{minipage}{0.4\textwidth}
		\begin{flushright}
            			    \large
			    \textit{Supervisor}\\
                                    Kostas \textsc{Kontogiannis}\\[0.5cm]
                            
                            \large
                \textit{Instructor}\\
                                    Nazim \textsc{Madhavji}\\
                            		\end{flushright}
	\end{minipage}
	
	% If you don't want a supervisor, uncomment the two lines below and comment the code above
	%{\large\textit{Author}}\\
	%John \textsc{Smith} % Your name

	%------------------------------------------------
	%	Date
	%------------------------------------------------

	\vfill\vfill % Position the date 3/4 down the remaining page
	
	{\large\today} % Date, change the \today to a set date if you want to be precise
	
	%------------------------------------------------
	%	Logo
	%------------------------------------------------
	
    	\vfill\vfill
	\includegraphics[width=0.2\textwidth]{../images/uwo.jpg}\\[1cm] % Include a department/university logo - this will require the graphicx package
    	 
	%----------------------------------------------------------------------------------------
	
	\vfill % Push the date up 1/4 of the remaining page
	
\end{titlepage}

\begin{abstract} 
{\bf Lorem ipsum sodales, accumsan neque eu, placerat purus. Interdum et
malesuada fames ac ante ipsum primis in faucibus. Nulla id varius metus,
id vestibulum purus. Nullam malesuada urna purus, quis euismod velit
tristique et. Fusce auctor laoreet arcu ac maximus. Duis ultricies
malesuada dui id pharetra. Donec tempus semper enim, in interdum ante
pharetra sed. Vivamus vel accumsan metus. Vivamus eu enim est. Duis ac
dolor a quam lacinia interdum in ut sem. Ut ipsum orci, dignissim vel
ante eget, blandit sollicitudin dolor.} Sed eu orci dolor sit amet, consectetur adipiscing elit. Duis dapibus
nisl vitae tempor placerat. Duis feugiat odio vitae quam pellentesque,
ac semper ex sagittis. Nunc id egestas tortor. Morbi nibh tortor,
suscipit vel libero quis, placerat molestie nulla. Nullam pellentesque
ex ac viverra lobortis. Donec hendrerit nibh nisi, a bibendum urna
efficitur ut. Cras venenatis sem magna, vel dignissim augue convallis a.
Proin sapien justo, viverra ac enim sit amet, cursus aliquet tellus.
Nulla at lacus magna. Nullam sit amet dui convallis, interdum felis eu,
viverra ligula. Pellentesque sed mollis nibh, at ultricies nisi.Quisque
id velit suscipit ipsum auctor egestas egestas sit amet dui. Curabitur
at sem nunc. Nunc non ultrices ex, et egestas odio. \end{abstract} 

\hypertarget{introduction-1.5-pages-max}{%
\section{\texorpdfstring{Introduction \emph{(1.5 pages
max)}}{Introduction (1.5 pages max)}}\label{introduction-1.5-pages-max}}

\hypertarget{background-and-related-work-2-pages-max}{%
\section{\texorpdfstring{Background and Related Work \emph{(2 pages
max)}}{Background and Related Work (2 pages max)}}\label{background-and-related-work-2-pages-max}}

This project was conceived from research that Professor Kontogainnis had
completed with past students regarding the task of determining faults in
a large codebase using only past bug reports. Past students had
experimented with multiple algorithms for processing large amounts of
bug reports in order to come up with an index of files that are most
likely to have a new bug within them.

The original system consisted of a collection of scripts written in
Java, Python and Shell. The scripts were fairly distributed and had very
little documentation. Most of the work was done in intermediate steps
and did not fit well together as a system because too much user
intervention was required between steps.

\hypertarget{concepts-terms-definitions-equations-1-page-max}{%
\section{\texorpdfstring{Concepts, Terms, Definitions, Equations
\emph{(1 page
max)}}{Concepts, Terms, Definitions, Equations (1 page max)}}\label{concepts-terms-definitions-equations-1-page-max}}

\hypertarget{rsf}{%
\subsubsection{RSF}\label{rsf}}

An RSF is a map of relationships betweens tokens within a codebase,
where a token is a keyword in the codebase such as a method. It is
generated using preprocessing scripts, and the result allows us to
verify tokens inside of bug reports.

\hypertarget{bug-report}{%
\subsubsection{Bug Report}\label{bug-report}}

A database entry for each bug report used for the analysis. Each bug
report contains a String field containing what a user wrote inside their
bug report.

\hypertarget{token-expansion}{%
\subsubsection{Token Expansion}\label{token-expansion}}

Tokens extracted from the bug report are expanded to find other similar
tokens. Token expansion includes tokens that are referenced by the
original token set.

\hypertarget{clique}{%
\subsubsection{Clique}\label{clique}}

\begin{itemize}
\tightlist
\item
  A collection of tokens that are referenced the most within the
  expanded set of tokens
\end{itemize}

\hypertarget{cluster}{%
\subsubsection{Cluster}\label{cluster}}

\begin{itemize}
\tightlist
\item
  A group of relationships that are closely related to each other
\end{itemize}

\hypertarget{development-objectives-0.75-page-max}{%
\section{\texorpdfstring{Development Objectives \emph{(0.75 page
max)}}{Development Objectives (0.75 page max)}}\label{development-objectives-0.75-page-max}}

\hypertarget{develop-a-complete-system-o1}{%
\subsubsection{Develop a complete system
(O1)}\label{develop-a-complete-system-o1}}

Since they are so many individual parts to the system required to run
the system, we wanted to combine scripts and Java code into an easy to
access complete system. Considering there were still some Python scripts
we did not get access to by the end of the project, it would be very
beneficial to have a complete system that is able to handle new
codebases and improve with new bug reports.

\hypertarget{be-able-to-integrate-with-a-github-based-workflow-o2}{%
\subsubsection{Be able to integrate with a GitHub based workflow
(O2)}\label{be-able-to-integrate-with-a-github-based-workflow-o2}}

We saw a great opportunity to integrate this system into GitHub's
system. Initially, the system was set up to read reports from BugZilla,
but we wanted to add the ability to integrate the system with GitHib's
issue tracker. This would entail fetching new issues as they are created
in a repo and processing them, Pushing results back to the issue page so
the developers can get a head start, and allowing users to authenticate
using GitHub.

\hypertarget{operate-as-a-standalone-service-with-a-ui-o3}{%
\subsubsection{Operate as a standalone service with a UI
(O3)}\label{operate-as-a-standalone-service-with-a-ui-o3}}

The system initially was just run through a terminal, and because there
aren't too many options needed from the user, it would be beneficial to
create a system that can work independantly with an easy to use
interface. This would Work in a similar fashion to other CI tools, such
as Travis. We would also plan to give users the control to hook into any
of their repositories

\hypertarget{system-requirements-2-pages-max}{%
\section{\texorpdfstring{System Requirements \emph{(2 pages
max)}}{System Requirements (2 pages max)}}\label{system-requirements-2-pages-max}}

\hypertarget{development-strategy-2-pages-max}{%
\section{\texorpdfstring{Development Strategy \emph{(2 pages
max)}}{Development Strategy (2 pages max)}}\label{development-strategy-2-pages-max}}

Since this project had a lot of the main functionality already
implemented, we did not have many choices over what development tools
and languages to use. The majority of the project was written in Java,
with some separate Python scripts used for data manipulation at some of
the stages in the system. Including the development tools previously
mentioned, the following were also used in the system:

\hypertarget{technologies}{%
\subsubsection{Technologies}\label{technologies}}

\begin{itemize}
\tightlist
\item
  Kotlin
\item
  Python
\item
  Java
\item
  Javalin
\item
  MongoDB
\item
  Github Webhooks
\end{itemize}

\hypertarget{tools}{%
\subsubsection{Tools}\label{tools}}

\begin{itemize}
\tightlist
\item
  Intellij Idea
\item
  NeoVim
\item
  Robo3T
\item
  ngrok
\end{itemize}

\hypertarget{datasets}{%
\subsection{Datasets}\label{datasets}}

\begin{itemize}
\tightlist
\item
  BugZilla reports
\end{itemize}

\hypertarget{results-10-pages-max}{%
\section{\texorpdfstring{Results \emph{(10 pages
max)}}{Results (10 pages max)}}\label{results-10-pages-max}}

\hypertarget{buglocalization-project}{%
\subsubsection{BugLocalization Project}\label{buglocalization-project}}

\begin{itemize}
\tightlist
\item
  Restructured and cleaned up codebase
\item
  Reduced very large codebase to 10 files
\item
  Documented and made more readable for easier integration
\end{itemize}

\hypertarget{api}{%
\subsubsection{API}\label{api}}

\begin{itemize}
\tightlist
\item
  Able to monitor a GitHub repo for events
\item
  Detect new issues and add into a Mongo database
\item
  Able to start and end deployments on Pull Requests
\end{itemize}

\hypertarget{discussion-1.5-pages-max}{%
\section{\texorpdfstring{Discussion \emph{(1.5 pages
max)}}{Discussion (1.5 pages max)}}\label{discussion-1.5-pages-max}}

\hypertarget{conclusions-1-page-max}{%
\section{\texorpdfstring{Conclusions \emph{(1 page
max)}}{Conclusions (1 page max)}}\label{conclusions-1-page-max}}

\hypertarget{late-start}{%
\subsubsection{Late Start}\label{late-start}}

\begin{itemize}
\tightlist
\item
  Second iteration of project
\item
  Had to re-purpose project after second milestone
\end{itemize}

\hypertarget{communication-problems}{%
\subsubsection{Communication problems}\label{communication-problems}}

\begin{itemize}
\tightlist
\item
  Lack of communication from the supervisor for long periods of time
\end{itemize}

\hypertarget{future-work-and-lessons-learnt-1-page-max}{%
\section{\texorpdfstring{Future Work and Lessons Learnt \emph{(1 page
max)}}{Future Work and Lessons Learnt (1 page max)}}\label{future-work-and-lessons-learnt-1-page-max}}

\hypertarget{future-work}{%
\subsubsection{Future work}\label{future-work}}

\begin{itemize}
\tightlist
\item
  Develop software to process issues incrementally into the
  BugLocalization
\item
  Integrate the BugLocalization project as part of the API and utilize
  its full power
\item
  Make a web UI similar to Travis to display results
\end{itemize}

\hypertarget{lessons-learnt}{%
\subsubsection{Lessons learnt}\label{lessons-learnt}}

\begin{itemize}
\tightlist
\item
  Develop consistent communication plan with stakeholders
\end{itemize}

\bibliography{}

%----------------------------------------------------------------------------------------

\end{document}
